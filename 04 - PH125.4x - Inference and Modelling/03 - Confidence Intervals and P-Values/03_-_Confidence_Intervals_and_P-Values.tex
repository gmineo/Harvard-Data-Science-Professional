\documentclass[]{article}
\usepackage{lmodern}
\usepackage{amssymb,amsmath}
\usepackage{ifxetex,ifluatex}
\usepackage{fixltx2e} % provides \textsubscript
\ifnum 0\ifxetex 1\fi\ifluatex 1\fi=0 % if pdftex
  \usepackage[T1]{fontenc}
  \usepackage[utf8]{inputenc}
\else % if luatex or xelatex
  \ifxetex
    \usepackage{mathspec}
  \else
    \usepackage{fontspec}
  \fi
  \defaultfontfeatures{Ligatures=TeX,Scale=MatchLowercase}
\fi
% use upquote if available, for straight quotes in verbatim environments
\IfFileExists{upquote.sty}{\usepackage{upquote}}{}
% use microtype if available
\IfFileExists{microtype.sty}{%
\usepackage{microtype}
\UseMicrotypeSet[protrusion]{basicmath} % disable protrusion for tt fonts
}{}
\usepackage[margin=1in]{geometry}
\usepackage{hyperref}
\hypersetup{unicode=true,
            pdftitle={Assessment 03 - Confidence Intervals},
            pdfauthor={Gabriele Mineo - Harvard Data Science Professional},
            pdfborder={0 0 0},
            breaklinks=true}
\urlstyle{same}  % don't use monospace font for urls
\usepackage{color}
\usepackage{fancyvrb}
\newcommand{\VerbBar}{|}
\newcommand{\VERB}{\Verb[commandchars=\\\{\}]}
\DefineVerbatimEnvironment{Highlighting}{Verbatim}{commandchars=\\\{\}}
% Add ',fontsize=\small' for more characters per line
\usepackage{framed}
\definecolor{shadecolor}{RGB}{248,248,248}
\newenvironment{Shaded}{\begin{snugshade}}{\end{snugshade}}
\newcommand{\KeywordTok}[1]{\textcolor[rgb]{0.13,0.29,0.53}{\textbf{#1}}}
\newcommand{\DataTypeTok}[1]{\textcolor[rgb]{0.13,0.29,0.53}{#1}}
\newcommand{\DecValTok}[1]{\textcolor[rgb]{0.00,0.00,0.81}{#1}}
\newcommand{\BaseNTok}[1]{\textcolor[rgb]{0.00,0.00,0.81}{#1}}
\newcommand{\FloatTok}[1]{\textcolor[rgb]{0.00,0.00,0.81}{#1}}
\newcommand{\ConstantTok}[1]{\textcolor[rgb]{0.00,0.00,0.00}{#1}}
\newcommand{\CharTok}[1]{\textcolor[rgb]{0.31,0.60,0.02}{#1}}
\newcommand{\SpecialCharTok}[1]{\textcolor[rgb]{0.00,0.00,0.00}{#1}}
\newcommand{\StringTok}[1]{\textcolor[rgb]{0.31,0.60,0.02}{#1}}
\newcommand{\VerbatimStringTok}[1]{\textcolor[rgb]{0.31,0.60,0.02}{#1}}
\newcommand{\SpecialStringTok}[1]{\textcolor[rgb]{0.31,0.60,0.02}{#1}}
\newcommand{\ImportTok}[1]{#1}
\newcommand{\CommentTok}[1]{\textcolor[rgb]{0.56,0.35,0.01}{\textit{#1}}}
\newcommand{\DocumentationTok}[1]{\textcolor[rgb]{0.56,0.35,0.01}{\textbf{\textit{#1}}}}
\newcommand{\AnnotationTok}[1]{\textcolor[rgb]{0.56,0.35,0.01}{\textbf{\textit{#1}}}}
\newcommand{\CommentVarTok}[1]{\textcolor[rgb]{0.56,0.35,0.01}{\textbf{\textit{#1}}}}
\newcommand{\OtherTok}[1]{\textcolor[rgb]{0.56,0.35,0.01}{#1}}
\newcommand{\FunctionTok}[1]{\textcolor[rgb]{0.00,0.00,0.00}{#1}}
\newcommand{\VariableTok}[1]{\textcolor[rgb]{0.00,0.00,0.00}{#1}}
\newcommand{\ControlFlowTok}[1]{\textcolor[rgb]{0.13,0.29,0.53}{\textbf{#1}}}
\newcommand{\OperatorTok}[1]{\textcolor[rgb]{0.81,0.36,0.00}{\textbf{#1}}}
\newcommand{\BuiltInTok}[1]{#1}
\newcommand{\ExtensionTok}[1]{#1}
\newcommand{\PreprocessorTok}[1]{\textcolor[rgb]{0.56,0.35,0.01}{\textit{#1}}}
\newcommand{\AttributeTok}[1]{\textcolor[rgb]{0.77,0.63,0.00}{#1}}
\newcommand{\RegionMarkerTok}[1]{#1}
\newcommand{\InformationTok}[1]{\textcolor[rgb]{0.56,0.35,0.01}{\textbf{\textit{#1}}}}
\newcommand{\WarningTok}[1]{\textcolor[rgb]{0.56,0.35,0.01}{\textbf{\textit{#1}}}}
\newcommand{\AlertTok}[1]{\textcolor[rgb]{0.94,0.16,0.16}{#1}}
\newcommand{\ErrorTok}[1]{\textcolor[rgb]{0.64,0.00,0.00}{\textbf{#1}}}
\newcommand{\NormalTok}[1]{#1}
\usepackage{graphicx,grffile}
\makeatletter
\def\maxwidth{\ifdim\Gin@nat@width>\linewidth\linewidth\else\Gin@nat@width\fi}
\def\maxheight{\ifdim\Gin@nat@height>\textheight\textheight\else\Gin@nat@height\fi}
\makeatother
% Scale images if necessary, so that they will not overflow the page
% margins by default, and it is still possible to overwrite the defaults
% using explicit options in \includegraphics[width, height, ...]{}
\setkeys{Gin}{width=\maxwidth,height=\maxheight,keepaspectratio}
\IfFileExists{parskip.sty}{%
\usepackage{parskip}
}{% else
\setlength{\parindent}{0pt}
\setlength{\parskip}{6pt plus 2pt minus 1pt}
}
\setlength{\emergencystretch}{3em}  % prevent overfull lines
\providecommand{\tightlist}{%
  \setlength{\itemsep}{0pt}\setlength{\parskip}{0pt}}
\setcounter{secnumdepth}{0}
% Redefines (sub)paragraphs to behave more like sections
\ifx\paragraph\undefined\else
\let\oldparagraph\paragraph
\renewcommand{\paragraph}[1]{\oldparagraph{#1}\mbox{}}
\fi
\ifx\subparagraph\undefined\else
\let\oldsubparagraph\subparagraph
\renewcommand{\subparagraph}[1]{\oldsubparagraph{#1}\mbox{}}
\fi

%%% Use protect on footnotes to avoid problems with footnotes in titles
\let\rmarkdownfootnote\footnote%
\def\footnote{\protect\rmarkdownfootnote}

%%% Change title format to be more compact
\usepackage{titling}

% Create subtitle command for use in maketitle
\newcommand{\subtitle}[1]{
  \posttitle{
    \begin{center}\large#1\end{center}
    }
}

\setlength{\droptitle}{-2em}

  \title{Assessment 03 - Confidence Intervals}
    \pretitle{\vspace{\droptitle}\centering\huge}
  \posttitle{\par}
    \author{Gabriele Mineo - Harvard Data Science Professional}
    \preauthor{\centering\large\emph}
  \postauthor{\par}
    \date{}
    \predate{}\postdate{}
  

\begin{document}
\maketitle

\subsection{\texorpdfstring{\textbf{Confidence interval for
p}}{Confidence interval for p}}\label{confidence-interval-for-p}

For the following exercises, we will use actual poll data from the 2016
election. The exercises will contain pre-loaded data from the dslabs
package.

\begin{Shaded}
\begin{Highlighting}[]
\KeywordTok{library}\NormalTok{(dslabs)}
\KeywordTok{library}\NormalTok{(dplyr)}
\end{Highlighting}
\end{Shaded}

\begin{verbatim}
## 
## Attaching package: 'dplyr'
\end{verbatim}

\begin{verbatim}
## The following objects are masked from 'package:stats':
## 
##     filter, lag
\end{verbatim}

\begin{verbatim}
## The following objects are masked from 'package:base':
## 
##     intersect, setdiff, setequal, union
\end{verbatim}

\begin{Shaded}
\begin{Highlighting}[]
\KeywordTok{library}\NormalTok{(ggplot2)}
\KeywordTok{data}\NormalTok{(polls_us_election_}\DecValTok{2016}\NormalTok{)}
\end{Highlighting}
\end{Shaded}

We will use all the national polls that ended within a few weeks before
the election.

Assume there are only two candidates and construct a 95\% confidence
interval for the election night proportion p.

Instructions

\begin{itemize}
\tightlist
\item
  Use \texttt{filter} to subset the data set for the poll data you want.
  Include polls that ended on or after October 31, 2016
  (\texttt{enddate}). Only include polls that took place in the United
  States. Call this filtered object \texttt{polls}.
\item
  Use \texttt{nrow} to make sure you created a filtered object
  \texttt{polls} that contains the correct number of rows.
\item
  Extract the sample size \texttt{N}from the first poll in your subset
  object \texttt{polls}.
\item
  Convert the percentage of Clinton voters (\texttt{rawpoll\_clinton})
  from the first poll in \texttt{polls} to a proportion,
  \texttt{X\_hat}. Print this value to the console.
\item
  Find the standard error of \texttt{X\_hat} given \texttt{N}. Print
  this result to the console.
\item
  Calculate the 95\% confidence interval of this estimate using the
  \texttt{qnorm} function.
\item
  Save the lower and upper confidence intervals as an object called
  \texttt{ci}. Save the lower confidence interval first.
\end{itemize}

\begin{Shaded}
\begin{Highlighting}[]
\CommentTok{# Load the data}

\KeywordTok{data}\NormalTok{(polls_us_election_}\DecValTok{2016}\NormalTok{)}

\CommentTok{# Generate an object `polls` that contains data filtered for polls that ended on or after October 31, 2016 in the United States}
\NormalTok{polls <-}\StringTok{ }\KeywordTok{filter}\NormalTok{(polls_us_election_}\DecValTok{2016}\NormalTok{, enddate }\OperatorTok{>=}\StringTok{ "2016-10-31"} \OperatorTok{&}\StringTok{ }\NormalTok{state }\OperatorTok{==}\StringTok{ "U.S."}\NormalTok{)}

\CommentTok{# How many rows does `polls` contain? Print this value to the console.}
\KeywordTok{nrow}\NormalTok{(polls)}
\end{Highlighting}
\end{Shaded}

\begin{verbatim}
## [1] 70
\end{verbatim}

\begin{Shaded}
\begin{Highlighting}[]
\CommentTok{# Assign the sample size of the first poll in `polls` to a variable called `N`. Print this value to the console.}
\NormalTok{N <-}\StringTok{ }\KeywordTok{head}\NormalTok{(polls}\OperatorTok{$}\NormalTok{samplesize,}\DecValTok{1}\NormalTok{)}
\NormalTok{N}
\end{Highlighting}
\end{Shaded}

\begin{verbatim}
## [1] 2220
\end{verbatim}

\begin{Shaded}
\begin{Highlighting}[]
\CommentTok{# For the first poll in `polls`, assign the estimated percentage of Clinton voters to a variable called `X_hat`. Print this value to the console.}
\NormalTok{X_hat <-}\StringTok{ }\NormalTok{(}\KeywordTok{head}\NormalTok{(polls}\OperatorTok{$}\NormalTok{rawpoll_clinton,}\DecValTok{1}\NormalTok{)}\OperatorTok{/}\DecValTok{100}\NormalTok{)}
\NormalTok{X_hat}
\end{Highlighting}
\end{Shaded}

\begin{verbatim}
## [1] 0.47
\end{verbatim}

\begin{Shaded}
\begin{Highlighting}[]
\CommentTok{# Calculate the standard error of `X_hat` and save it to a variable called `se_hat`. Print this value to the console.}
\NormalTok{se_hat <-}\StringTok{ }\KeywordTok{sqrt}\NormalTok{(X_hat}\OperatorTok{*}\NormalTok{(}\DecValTok{1}\OperatorTok{-}\NormalTok{X_hat)}\OperatorTok{/}\NormalTok{N)}
\NormalTok{se_hat}
\end{Highlighting}
\end{Shaded}

\begin{verbatim}
## [1] 0.01059279
\end{verbatim}

\begin{Shaded}
\begin{Highlighting}[]
\CommentTok{# Use `qnorm` to calculate the 95% confidence interval for the proportion of Clinton voters. Save the lower and then the upper confidence interval to a variable called `ci`.}
\KeywordTok{qnorm}\NormalTok{(}\FloatTok{0.975}\NormalTok{)}
\end{Highlighting}
\end{Shaded}

\begin{verbatim}
## [1] 1.959964
\end{verbatim}

\begin{Shaded}
\begin{Highlighting}[]
\NormalTok{ci <-}\StringTok{ }\KeywordTok{c}\NormalTok{(X_hat }\OperatorTok{-}\StringTok{ }\KeywordTok{qnorm}\NormalTok{(}\FloatTok{0.975}\NormalTok{)}\OperatorTok{*}\NormalTok{se_hat, X_hat }\OperatorTok{+}\StringTok{ }\KeywordTok{qnorm}\NormalTok{(}\FloatTok{0.975}\NormalTok{)}\OperatorTok{*}\NormalTok{se_hat)}
\end{Highlighting}
\end{Shaded}

\subsection{\texorpdfstring{\textbf{Pollster results for
p}}{Pollster results for p}}\label{pollster-results-for-p}

Create a new object called \texttt{pollster\_results} that contains the
pollster's name, the end date of the poll, the proportion of voters who
declared a vote for Clinton, the standard error of this estimate, and
the lower and upper bounds of the confidence interval for the estimate.

Instructions

\begin{itemize}
\tightlist
\item
  Use the \texttt{mutate} function to define four new columns:
  \texttt{X\_hat,\ se\_hat,\ lower}, and \texttt{upper}. Temporarily add
  these columns to the \texttt{polls} object that has already been
  loaded for you.
\item
  In the \texttt{X\_hat}column, convert the raw poll results for Clinton
  to a proportion.
\item
  In the \texttt{se\_hat} column, calculate the standard error of
  \texttt{X\_hat} for each poll using the \texttt{sqrt} function.
\item
  In the \texttt{lower} column, calculate the lower bound of the 95\%
  confidence interval using the \texttt{qnorm} function.
\item
  In the \texttt{upper} column, calculate the upper bound of the 95\%
  confidence interval using the \texttt{qnorm} function.
\item
  Use the \texttt{select} function to select the columns from
  \texttt{polls} to save to the new object \texttt{pollster\_results}.
\end{itemize}

\begin{Shaded}
\begin{Highlighting}[]
\CommentTok{# The `polls` object that filtered all the data by date and nation has already been loaded. Examine it using the `head` function.}
\KeywordTok{head}\NormalTok{(polls)}
\end{Highlighting}
\end{Shaded}

\begin{verbatim}
##   state  startdate    enddate
## 1  U.S. 2016-11-03 2016-11-06
## 2  U.S. 2016-11-01 2016-11-07
## 3  U.S. 2016-11-02 2016-11-06
## 4  U.S. 2016-11-04 2016-11-07
## 5  U.S. 2016-11-03 2016-11-06
## 6  U.S. 2016-11-03 2016-11-06
##                                                     pollster grade
## 1                                   ABC News/Washington Post    A+
## 2                                    Google Consumer Surveys     B
## 3                                                      Ipsos    A-
## 4                                                     YouGov     B
## 5                                           Gravis Marketing    B-
## 6 Fox News/Anderson Robbins Research/Shaw & Company Research     A
##   samplesize population rawpoll_clinton rawpoll_trump rawpoll_johnson
## 1       2220         lv           47.00         43.00            4.00
## 2      26574         lv           38.03         35.69            5.46
## 3       2195         lv           42.00         39.00            6.00
## 4       3677         lv           45.00         41.00            5.00
## 5      16639         rv           47.00         43.00            3.00
## 6       1295         lv           48.00         44.00            3.00
##   rawpoll_mcmullin adjpoll_clinton adjpoll_trump adjpoll_johnson
## 1               NA        45.20163      41.72430        4.626221
## 2               NA        43.34557      41.21439        5.175792
## 3               NA        42.02638      38.81620        6.844734
## 4               NA        45.65676      40.92004        6.069454
## 5               NA        46.84089      42.33184        3.726098
## 6               NA        49.02208      43.95631        3.057876
##   adjpoll_mcmullin
## 1               NA
## 2               NA
## 3               NA
## 4               NA
## 5               NA
## 6               NA
\end{verbatim}

\begin{Shaded}
\begin{Highlighting}[]
\CommentTok{# Create a new object called `pollster_results` that contains columns for pollster name, end date, X_hat, lower confidence interval, and upper confidence interval for each poll.}
\NormalTok{polls <-}\StringTok{ }\KeywordTok{mutate}\NormalTok{(polls, }\DataTypeTok{X_hat =}\NormalTok{ polls}\OperatorTok{$}\NormalTok{rawpoll_clinton}\OperatorTok{/}\DecValTok{100}\NormalTok{, }\DataTypeTok{se_hat =} \KeywordTok{sqrt}\NormalTok{(X_hat}\OperatorTok{*}\NormalTok{(}\DecValTok{1}\OperatorTok{-}\NormalTok{X_hat)}\OperatorTok{/}\NormalTok{polls}\OperatorTok{$}\NormalTok{samplesize), }\DataTypeTok{lower =}\NormalTok{ X_hat }\OperatorTok{-}\StringTok{ }\KeywordTok{qnorm}\NormalTok{(}\FloatTok{0.975}\NormalTok{)}\OperatorTok{*}\NormalTok{se_hat, }\DataTypeTok{upper =}\NormalTok{ X_hat }\OperatorTok{+}\StringTok{ }\KeywordTok{qnorm}\NormalTok{(}\FloatTok{0.975}\NormalTok{)}\OperatorTok{*}\NormalTok{se_hat)}
\NormalTok{pollster_results <-}\StringTok{ }\KeywordTok{select}\NormalTok{(polls, pollster, enddate, X_hat, se_hat, lower, upper)}
\end{Highlighting}
\end{Shaded}

\subsection{\texorpdfstring{\textbf{Comparing to actual results -
p}}{Comparing to actual results - p}}\label{comparing-to-actual-results---p}

The final tally for the popular vote was Clinton 48.2\% and Trump
46.1\%. Add a column called \texttt{hit} to \texttt{pollster\_results}
that states if the confidence interval included the true proportion
p=0.482 or not. What proportion of confidence intervals included p?

Instructions

\begin{itemize}
\tightlist
\item
  Use the \texttt{mutate} function to define a new variable called
  `hit'.
\item
  Use logical expressions to determine if each values in \texttt{lower}
  and \texttt{upper} span the actual proportion.
\item
  Use the \texttt{mean} function to determine the average value in
  \texttt{hit} and \texttt{summarize} the results using summarize.
\item
  Save the result as an object called \texttt{avg\_hit}.
\end{itemize}

\begin{Shaded}
\begin{Highlighting}[]
\CommentTok{# The `pollster_results` object has already been loaded. Examine it using the `head` function.}
\KeywordTok{head}\NormalTok{(pollster_results)}
\end{Highlighting}
\end{Shaded}

\begin{verbatim}
##                                                     pollster    enddate
## 1                                   ABC News/Washington Post 2016-11-06
## 2                                    Google Consumer Surveys 2016-11-07
## 3                                                      Ipsos 2016-11-06
## 4                                                     YouGov 2016-11-07
## 5                                           Gravis Marketing 2016-11-06
## 6 Fox News/Anderson Robbins Research/Shaw & Company Research 2016-11-06
##    X_hat      se_hat     lower     upper
## 1 0.4700 0.010592790 0.4492385 0.4907615
## 2 0.3803 0.002978005 0.3744632 0.3861368
## 3 0.4200 0.010534681 0.3993524 0.4406476
## 4 0.4500 0.008204286 0.4339199 0.4660801
## 5 0.4700 0.003869218 0.4624165 0.4775835
## 6 0.4800 0.013883131 0.4527896 0.5072104
\end{verbatim}

\begin{Shaded}
\begin{Highlighting}[]
\CommentTok{# Add a logical variable called `hit` that indicates whether the actual value exists within the confidence interval of each poll. Summarize the average `hit` result to determine the proportion of polls with confidence intervals include the actual value. Save the result as an object called `avg_hit`.}
\NormalTok{avg_hit <-}\StringTok{ }\NormalTok{pollster_results }\OperatorTok\StringTok{ }\KeywordTok{mutate}\NormalTok{(}\DataTypeTok{hit=}\NormalTok{(lower}\OperatorTok{<}\FloatTok{0.482} \OperatorTok{&}\StringTok{ }\NormalTok{upper}\OperatorTok{>}\FloatTok{0.482}\NormalTok{)) }\OperatorTok\StringTok{ }\KeywordTok{summarize}\NormalTok{(}\KeywordTok{mean}\NormalTok{(hit))}
\NormalTok{avg_hit}
\end{Highlighting}
\end{Shaded}

\begin{verbatim}
##   mean(hit)
## 1 0.3142857
\end{verbatim}

\subsection{\texorpdfstring{\textbf{Theory of confidence
intervals}}{Theory of confidence intervals}}\label{theory-of-confidence-intervals}

If these confidence intervals are constructed correctly, and the theory
holds up, what proportion of confidence intervals should include p?

Possible Answers

\begin{itemize}
\tightlist
\item
  0.05
\item
  0.31
\item
  0.50
\item
  0.95 {[}X{]}
\end{itemize}

\subsection{\texorpdfstring{\textbf{Confidence interval for
d}}{Confidence interval for d}}\label{confidence-interval-for-d}

A much smaller proportion of the polls than expected produce confidence
intervals containing p.~Notice that most polls that fail to include p
are underestimating. The rationale for this is that undecided voters
historically divide evenly between the two main candidates on election
day.

In this case, it is more informative to estimate the spread or the
difference between the proportion of two candidates d, or
0.482−0.461=0.021 for this election.

Assume that there are only two parties and that \(\ d=2p−1\). Construct
a 95\% confidence interval for difference in proportions on election
night.

Instructions

\begin{itemize}
\tightlist
\item
  Use the \texttt{mutate} function to define a new variable called
  `d\_hat' in \texttt{polls}. The new variable subtract the proportion
  of Trump voters from the proportion of Clinton voters.
\item
  Extract the sample size \texttt{N} from the first poll in your subset
  object \texttt{polls}.
\item
  Extract the difference in proportions of voters \texttt{d\_hat} from
  the first poll in your subset object polls.
\item
  Use the formula above to calculate p from \texttt{d\_hat}. Assign p to
  the variable \texttt{X\_hat}.
\item
  Find the standard error of the spread given \texttt{N}.
\item
  Calculate the 95\% confidence interval of this estimate of the
  difference in proportions, \texttt{d\_hat}, using the \texttt{qnorm}
  function.
\item
  Save the lower and upper confidence intervals as an object called
  \texttt{ci}. Save the lower confidence interval first.
\end{itemize}

\begin{Shaded}
\begin{Highlighting}[]
\CommentTok{# Add a statement to this line of code that will add a new column named `d_hat` to `polls`. The new column should contain the difference in the proportion of voters.}
\NormalTok{polls <-}\StringTok{ }\NormalTok{polls_us_election_}\DecValTok{2016} \OperatorTok\StringTok{ }\KeywordTok{filter}\NormalTok{(enddate }\OperatorTok{>=}\StringTok{ "2016-10-31"} \OperatorTok{&}\StringTok{ }\NormalTok{state }\OperatorTok{==}\StringTok{ "U.S."}\NormalTok{)  }\OperatorTok
\StringTok{  }\KeywordTok{mutate}\NormalTok{(}\DataTypeTok{d_hat =}\NormalTok{ rawpoll_clinton}\OperatorTok{/}\DecValTok{100} \OperatorTok{-}\StringTok{ }\NormalTok{rawpoll_trump}\OperatorTok{/}\DecValTok{100}\NormalTok{)}

\CommentTok{# Assign the sample size of the first poll in `polls` to a variable called `N`. Print this value to the console.}
\NormalTok{N <-}\StringTok{ }\NormalTok{polls}\OperatorTok{$}\NormalTok{samplesize[}\DecValTok{1}\NormalTok{]}

\CommentTok{# For the difference `d_hat` of the first poll in `polls` to a variable called `d_hat`. Print this value to the console.}
\NormalTok{d_hat <-}\StringTok{ }\NormalTok{polls}\OperatorTok{$}\NormalTok{d_hat[}\DecValTok{1}\NormalTok{]}
\NormalTok{d_hat}
\end{Highlighting}
\end{Shaded}

\begin{verbatim}
## [1] 0.04
\end{verbatim}

\begin{Shaded}
\begin{Highlighting}[]
\CommentTok{# Assign proportion of votes for Clinton to the variable `X_hat`.}
\NormalTok{X_hat <-}\StringTok{ }\NormalTok{(d_hat}\OperatorTok{+}\DecValTok{1}\NormalTok{)}\OperatorTok{/}\DecValTok{2}

\CommentTok{# Calculate the standard error of the spread and save it to a variable called `se_hat`. Print this value to the console.}
\NormalTok{se_hat <-}\StringTok{ }\DecValTok{2}\OperatorTok{*}\KeywordTok{sqrt}\NormalTok{(X_hat}\OperatorTok{*}\NormalTok{(}\DecValTok{1}\OperatorTok{-}\NormalTok{X_hat)}\OperatorTok{/}\NormalTok{N)}
\NormalTok{se_hat}
\end{Highlighting}
\end{Shaded}

\begin{verbatim}
## [1] 0.02120683
\end{verbatim}

\begin{Shaded}
\begin{Highlighting}[]
\CommentTok{# Use `qnorm` to calculate the 95% confidence interval for the difference in the proportions of voters. Save the lower and then the upper confidence interval to a variable called `ci`.}
\NormalTok{ci <-}\StringTok{ }\KeywordTok{c}\NormalTok{(d_hat }\OperatorTok{-}\StringTok{ }\KeywordTok{qnorm}\NormalTok{(}\FloatTok{0.975}\NormalTok{)}\OperatorTok{*}\NormalTok{se_hat, d_hat }\OperatorTok{+}\StringTok{ }\KeywordTok{qnorm}\NormalTok{(}\FloatTok{0.975}\NormalTok{)}\OperatorTok{*}\NormalTok{se_hat)}
\end{Highlighting}
\end{Shaded}

\subsection{\texorpdfstring{\textbf{Pollster results for
d}}{Pollster results for d}}\label{pollster-results-for-d}

Create a new object called \texttt{pollster\_results} that contains the
pollster's name, the end date of the poll, the difference in the
proportion of voters who declared a vote either, the standard error of
this estimate, and the lower and upper bounds of the confidence interval
for the estimate.

Instructions

\begin{itemize}
\tightlist
\item
  Use the \texttt{mutate} function to define four new columns: `X\_hat',
  `se\_hat', `lower', and `upper'. Temporarily add these columns to the
  polls object that has already been loaded for you.
\item
  In the \texttt{X\_hat} column, calculate the proportion of voters for
  Clinton using \texttt{d\_hat}.
\item
  In the \texttt{se\_hat} column, calculate the standard error of the
  spread for each poll using the \texttt{sqrt}function.
\item
  In the l\texttt{ower} column, calculate the lower bound of the 95\%
  confidence interval using the \texttt{qnorm} function.
\item
  In the \texttt{upper} column, calculate the upper bound of the 95\%
  confidence interval using the \texttt{qnorm} function.
\item
  Use the \texttt{select} function to select the columns from
  \texttt{polls} to save to the new object \texttt{pollster\_results}.
\end{itemize}

\begin{Shaded}
\begin{Highlighting}[]
\CommentTok{# The subset `polls` data with 'd_hat' already calculated has been loaded. Examine it using the `head` function.}
\KeywordTok{head}\NormalTok{(polls)}
\end{Highlighting}
\end{Shaded}

\begin{verbatim}
##   state  startdate    enddate
## 1  U.S. 2016-11-03 2016-11-06
## 2  U.S. 2016-11-01 2016-11-07
## 3  U.S. 2016-11-02 2016-11-06
## 4  U.S. 2016-11-04 2016-11-07
## 5  U.S. 2016-11-03 2016-11-06
## 6  U.S. 2016-11-03 2016-11-06
##                                                     pollster grade
## 1                                   ABC News/Washington Post    A+
## 2                                    Google Consumer Surveys     B
## 3                                                      Ipsos    A-
## 4                                                     YouGov     B
## 5                                           Gravis Marketing    B-
## 6 Fox News/Anderson Robbins Research/Shaw & Company Research     A
##   samplesize population rawpoll_clinton rawpoll_trump rawpoll_johnson
## 1       2220         lv           47.00         43.00            4.00
## 2      26574         lv           38.03         35.69            5.46
## 3       2195         lv           42.00         39.00            6.00
## 4       3677         lv           45.00         41.00            5.00
## 5      16639         rv           47.00         43.00            3.00
## 6       1295         lv           48.00         44.00            3.00
##   rawpoll_mcmullin adjpoll_clinton adjpoll_trump adjpoll_johnson
## 1               NA        45.20163      41.72430        4.626221
## 2               NA        43.34557      41.21439        5.175792
## 3               NA        42.02638      38.81620        6.844734
## 4               NA        45.65676      40.92004        6.069454
## 5               NA        46.84089      42.33184        3.726098
## 6               NA        49.02208      43.95631        3.057876
##   adjpoll_mcmullin  d_hat
## 1               NA 0.0400
## 2               NA 0.0234
## 3               NA 0.0300
## 4               NA 0.0400
## 5               NA 0.0400
## 6               NA 0.0400
\end{verbatim}

\begin{Shaded}
\begin{Highlighting}[]
\CommentTok{# Create a new object called `pollster_results` that contains columns for pollster name, end date, d_hat, lower confidence interval of d_hat, and upper confidence interval of d_hat for each poll.}
\NormalTok{pollster_results <-}\StringTok{ }\NormalTok{polls }\OperatorTok\StringTok{ }\KeywordTok{mutate}\NormalTok{(}\DataTypeTok{X_hat =}\NormalTok{ (d_hat }\OperatorTok{+}\StringTok{ }\DecValTok{1}\NormalTok{) }\OperatorTok{/}\StringTok{ }\DecValTok{2}\NormalTok{) }\OperatorTok\StringTok{ }\KeywordTok{mutate}\NormalTok{(}\DataTypeTok{se_hat =} \DecValTok{2} \OperatorTok{*}\StringTok{ }\KeywordTok{sqrt}\NormalTok{(X_hat }\OperatorTok{*}\StringTok{ }\NormalTok{(}\DecValTok{1} \OperatorTok{-}\StringTok{ }\NormalTok{X_hat) }\OperatorTok{/}\StringTok{ }\NormalTok{samplesize)) }\OperatorTok\StringTok{ }\KeywordTok{mutate}\NormalTok{(}\DataTypeTok{lower =}\NormalTok{ d_hat }\OperatorTok{-}\StringTok{ }\KeywordTok{qnorm}\NormalTok{(}\FloatTok{0.975}\NormalTok{) }\OperatorTok{*}\StringTok{ }\NormalTok{se_hat) }\OperatorTok\StringTok{ }\KeywordTok{mutate}\NormalTok{(}\DataTypeTok{upper =}\NormalTok{ d_hat }\OperatorTok{+}\StringTok{ }\KeywordTok{qnorm}\NormalTok{(}\FloatTok{0.975}\NormalTok{) }\OperatorTok{*}\StringTok{ }\NormalTok{se_hat) }\OperatorTok\StringTok{ }\KeywordTok{select}\NormalTok{(pollster, enddate, d_hat, lower, upper)}
\NormalTok{pollster_results}
\end{Highlighting}
\end{Shaded}

\begin{verbatim}
##                                                      pollster    enddate
## 1                                    ABC News/Washington Post 2016-11-06
## 2                                     Google Consumer Surveys 2016-11-07
## 3                                                       Ipsos 2016-11-06
## 4                                                      YouGov 2016-11-07
## 5                                            Gravis Marketing 2016-11-06
## 6  Fox News/Anderson Robbins Research/Shaw & Company Research 2016-11-06
## 7                                     CBS News/New York Times 2016-11-06
## 8                                NBC News/Wall Street Journal 2016-11-05
## 9                                                    IBD/TIPP 2016-11-07
## 10                                           Selzer & Company 2016-11-06
## 11                                          Angus Reid Global 2016-11-04
## 12                                        Monmouth University 2016-11-06
## 13                                             Marist College 2016-11-03
## 14                                   The Times-Picayune/Lucid 2016-11-07
## 15                                      USC Dornsife/LA Times 2016-11-07
## 16                      RKM Research and Communications, Inc. 2016-11-05
## 17                                       CVOTER International 2016-11-06
## 18                                            Morning Consult 2016-11-05
## 19                                               SurveyMonkey 2016-11-06
## 20                   Rasmussen Reports/Pulse Opinion Research 2016-11-06
## 21                                              Insights West 2016-11-07
## 22                                 RAND (American Life Panel) 2016-11-01
## 23 Fox News/Anderson Robbins Research/Shaw & Company Research 2016-11-03
## 24                                    CBS News/New York Times 2016-11-01
## 25                                   ABC News/Washington Post 2016-11-05
## 26                                                      Ipsos 2016-11-04
## 27                                   ABC News/Washington Post 2016-11-04
## 28                                                     YouGov 2016-11-06
## 29                                                   IBD/TIPP 2016-11-06
## 30                                   ABC News/Washington Post 2016-11-03
## 31                                                   IBD/TIPP 2016-11-03
## 32                                                   IBD/TIPP 2016-11-05
## 33                                   ABC News/Washington Post 2016-11-02
## 34                                   ABC News/Washington Post 2016-11-01
## 35                                   ABC News/Washington Post 2016-10-31
## 36                                                      Ipsos 2016-11-03
## 37                                                   IBD/TIPP 2016-11-04
## 38                                                     YouGov 2016-11-01
## 39                                                   IBD/TIPP 2016-10-31
## 40                                                      Ipsos 2016-11-02
## 41                   Rasmussen Reports/Pulse Opinion Research 2016-11-03
## 42                                   The Times-Picayune/Lucid 2016-11-06
## 43                                                      Ipsos 2016-11-01
## 44                                                   IBD/TIPP 2016-11-02
## 45                                                   IBD/TIPP 2016-11-01
## 46                   Rasmussen Reports/Pulse Opinion Research 2016-11-02
## 47                                                      Ipsos 2016-10-31
## 48                                   The Times-Picayune/Lucid 2016-11-05
## 49                   Rasmussen Reports/Pulse Opinion Research 2016-10-31
## 50                                    Google Consumer Surveys 2016-10-31
## 51                                       CVOTER International 2016-11-05
## 52                   Rasmussen Reports/Pulse Opinion Research 2016-11-01
## 53                                       CVOTER International 2016-11-04
## 54                                   The Times-Picayune/Lucid 2016-11-04
## 55                                      USC Dornsife/LA Times 2016-11-06
## 56                                       CVOTER International 2016-11-03
## 57                                   The Times-Picayune/Lucid 2016-11-03
## 58                                      USC Dornsife/LA Times 2016-11-05
## 59                                       CVOTER International 2016-11-02
## 60                                      USC Dornsife/LA Times 2016-11-04
## 61                                       CVOTER International 2016-11-01
## 62                                   The Times-Picayune/Lucid 2016-11-02
## 63                                           Gravis Marketing 2016-10-31
## 64                                      USC Dornsife/LA Times 2016-11-03
## 65                                   The Times-Picayune/Lucid 2016-11-01
## 66                                      USC Dornsife/LA Times 2016-11-02
## 67                                           Gravis Marketing 2016-11-02
## 68                                      USC Dornsife/LA Times 2016-11-01
## 69                                   The Times-Picayune/Lucid 2016-10-31
## 70                                      USC Dornsife/LA Times 2016-10-31
##      d_hat        lower         upper
## 1   0.0400 -0.001564627  0.0815646272
## 2   0.0234  0.011380104  0.0354198955
## 3   0.0300 -0.011815309  0.0718153088
## 4   0.0400  0.007703641  0.0722963589
## 5   0.0400  0.024817728  0.0551822719
## 6   0.0400 -0.014420872  0.0944208716
## 7   0.0400 -0.011860967  0.0918609675
## 8   0.0400 -0.014696100  0.0946961005
## 9  -0.0150 -0.073901373  0.0439013728
## 10  0.0300 -0.039307332  0.0993073320
## 11  0.0400 -0.017724837  0.0977248370
## 12  0.0600 -0.011534270  0.1315342703
## 13  0.0100 -0.053923780  0.0739237800
## 14  0.0500  0.011013152  0.0889868476
## 15 -0.0323 -0.068233293  0.0036332933
## 16  0.0320 -0.029670864  0.0936708643
## 17  0.0278 -0.020801931  0.0764019309
## 18  0.0300 -0.020889533  0.0808895334
## 19  0.0600  0.052615604  0.0673843956
## 20  0.0200 -0.030595930  0.0705959303
## 21  0.0400 -0.023875814  0.1038758144
## 22  0.0910  0.050024415  0.1319755848
## 23  0.0150 -0.043901373  0.0739013728
## 24  0.0300 -0.034344689  0.0943446886
## 25  0.0400 -0.004497495  0.0844974954
## 26  0.0400 -0.001341759  0.0813417590
## 27  0.0500  0.002312496  0.0976875039
## 28  0.0390  0.032254341  0.0457456589
## 29 -0.0240 -0.085171524  0.0371715236
## 30  0.0400 -0.011988727  0.0919887265
## 31  0.0050 -0.060404028  0.0704040277
## 32 -0.0100 -0.075220256  0.0552202561
## 33  0.0300 -0.027745070  0.0877450695
## 34  0.0200 -0.037362198  0.0773621983
## 35  0.0000 -0.057008466  0.0570084657
## 36  0.0490  0.005454535  0.0925454654
## 37  0.0050 -0.064121736  0.0741217361
## 38  0.0300 -0.025791885  0.0857918847
## 39  0.0090 -0.052426619  0.0704266191
## 40  0.0820  0.037443982  0.1265560180
## 41  0.0000 -0.050606052  0.0506060525
## 42  0.0500  0.011491350  0.0885086495
## 43  0.0730  0.026563876  0.1194361238
## 44 -0.0010 -0.067563833  0.0655638334
## 45 -0.0040 -0.070756104  0.0627561042
## 46 -0.0300 -0.080583275  0.0205832746
## 47  0.0680  0.016894089  0.1191059111
## 48  0.0500  0.011051757  0.0889482429
## 49  0.0000 -0.050606052  0.0506060525
## 50  0.0262  0.013635277  0.0387647229
## 51  0.0333 -0.016106137  0.0827061365
## 52  0.0000 -0.050606052  0.0506060525
## 53  0.0124 -0.038560140  0.0633601401
## 54  0.0600  0.021340150  0.0986598497
## 55 -0.0475 -0.083637121 -0.0113628793
## 56  0.0009 -0.051576011  0.0533760106
## 57  0.0500  0.011807815  0.0881921851
## 58 -0.0553 -0.091100799 -0.0194992008
## 59  0.0056 -0.048162418  0.0593624177
## 60 -0.0540 -0.089809343 -0.0181906570
## 61  0.0067 -0.046002019  0.0594020190
## 62  0.0500  0.011756829  0.0882431712
## 63  0.0100 -0.016769729  0.0367697294
## 64 -0.0351 -0.071090499  0.0008904993
## 65  0.0300 -0.008295761  0.0682957614
## 66 -0.0503 -0.086413709 -0.0141862908
## 67  0.0200 -0.019711083  0.0597110831
## 68 -0.0547 -0.090406512 -0.0189934879
## 69  0.0200 -0.018430368  0.0584303678
## 70 -0.0362 -0.071126335 -0.0012736645
\end{verbatim}

\subsection{\texorpdfstring{\textbf{Comparing to actual results -
d}}{Comparing to actual results - d}}\label{comparing-to-actual-results---d}

What proportion of confidence intervals for the difference between the
proportion of voters included d, the actual difference in election day?

Instructions

\begin{itemize}
\tightlist
\item
  Use the \texttt{mutate} function to define a new variable
  within\texttt{pollster\_results} called hit.
\item
  Use logical expressions to determine if each values in \texttt{lower}
  and \texttt{upper} span the actual difference in proportions of
  voters.
\item
  Use the \texttt{mean} function to determine the average value in
  \texttt{hit} and summarize the results using \texttt{summarize}.
\item
  Save the result as an object called \texttt{avg\_hit}.
\end{itemize}

\begin{Shaded}
\begin{Highlighting}[]
\CommentTok{# The `pollster_results` object has already been loaded. Examine it using the `head` function.}
\KeywordTok{head}\NormalTok{(pollster_results)}
\end{Highlighting}
\end{Shaded}

\begin{verbatim}
##                                                     pollster    enddate
## 1                                   ABC News/Washington Post 2016-11-06
## 2                                    Google Consumer Surveys 2016-11-07
## 3                                                      Ipsos 2016-11-06
## 4                                                     YouGov 2016-11-07
## 5                                           Gravis Marketing 2016-11-06
## 6 Fox News/Anderson Robbins Research/Shaw & Company Research 2016-11-06
##    d_hat        lower      upper
## 1 0.0400 -0.001564627 0.08156463
## 2 0.0234  0.011380104 0.03541990
## 3 0.0300 -0.011815309 0.07181531
## 4 0.0400  0.007703641 0.07229636
## 5 0.0400  0.024817728 0.05518227
## 6 0.0400 -0.014420872 0.09442087
\end{verbatim}

\begin{Shaded}
\begin{Highlighting}[]
\CommentTok{# Add a logical variable called `hit` that indicates whether the actual value (0.021) exists within the confidence interval of each poll. Summarize the average `hit` result to determine the proportion of polls with confidence intervals include the actual value. Save the result as an object called `avg_hit`.}
\NormalTok{avg_hit <-}\StringTok{ }\NormalTok{pollster_results }\OperatorTok\StringTok{ }\KeywordTok{mutate}\NormalTok{(}\DataTypeTok{hit=}\NormalTok{lower }\OperatorTok{<=}\StringTok{ }\FloatTok{0.021} \OperatorTok{&}\StringTok{ }\NormalTok{upper }\OperatorTok{>=}\StringTok{ }\FloatTok{0.021}\NormalTok{) }\OperatorTok\StringTok{ }\KeywordTok{summarize}\NormalTok{(}\KeywordTok{mean}\NormalTok{(hit))}
\end{Highlighting}
\end{Shaded}

\subsection{\texorpdfstring{\textbf{Comparing to actual results by
pollster}}{Comparing to actual results by pollster}}\label{comparing-to-actual-results-by-pollster}

Although the proportion of confidence intervals that include the actual
difference between the proportion of voters increases substantially, it
is still lower that 0.95. In the next chapter, we learn the reason for
this.

To motivate our next exercises, calculate the difference between each
poll's estimate d¯ and the actual d=0.021. Stratify this difference, or
error, by pollster in a plot.

Instructions

\begin{itemize}
\tightlist
\item
  Define a new variable \texttt{errors} that contains the difference
  between the estimated difference between the proportion of voters and
  the actual difference on election day, 0.021.
\item
  To create the plot of errors by pollster, add a layer with the
  function \texttt{geom\_point}. The aesthetic mappings require a
  definition of the x-axis and y-axis variables. So the code looks like
  the example below, but you fill in the variables for x and y.
\item
  The last line of the example code adjusts the x-axis labels so that
  they are easier to read.
\end{itemize}

\begin{verbatim}
data %>% ggplot(aes(x = , y = )) +
  geom_point() +
  theme(axis.text.x = element_text(angle = 90, hjust = 1))
\end{verbatim}

\begin{Shaded}
\begin{Highlighting}[]
\CommentTok{# The `polls` object has already been loaded. Examine it using the `head` function.}
\KeywordTok{head}\NormalTok{(polls)}
\end{Highlighting}
\end{Shaded}

\begin{verbatim}
##   state  startdate    enddate
## 1  U.S. 2016-11-03 2016-11-06
## 2  U.S. 2016-11-01 2016-11-07
## 3  U.S. 2016-11-02 2016-11-06
## 4  U.S. 2016-11-04 2016-11-07
## 5  U.S. 2016-11-03 2016-11-06
## 6  U.S. 2016-11-03 2016-11-06
##                                                     pollster grade
## 1                                   ABC News/Washington Post    A+
## 2                                    Google Consumer Surveys     B
## 3                                                      Ipsos    A-
## 4                                                     YouGov     B
## 5                                           Gravis Marketing    B-
## 6 Fox News/Anderson Robbins Research/Shaw & Company Research     A
##   samplesize population rawpoll_clinton rawpoll_trump rawpoll_johnson
## 1       2220         lv           47.00         43.00            4.00
## 2      26574         lv           38.03         35.69            5.46
## 3       2195         lv           42.00         39.00            6.00
## 4       3677         lv           45.00         41.00            5.00
## 5      16639         rv           47.00         43.00            3.00
## 6       1295         lv           48.00         44.00            3.00
##   rawpoll_mcmullin adjpoll_clinton adjpoll_trump adjpoll_johnson
## 1               NA        45.20163      41.72430        4.626221
## 2               NA        43.34557      41.21439        5.175792
## 3               NA        42.02638      38.81620        6.844734
## 4               NA        45.65676      40.92004        6.069454
## 5               NA        46.84089      42.33184        3.726098
## 6               NA        49.02208      43.95631        3.057876
##   adjpoll_mcmullin  d_hat
## 1               NA 0.0400
## 2               NA 0.0234
## 3               NA 0.0300
## 4               NA 0.0400
## 5               NA 0.0400
## 6               NA 0.0400
\end{verbatim}

\begin{Shaded}
\begin{Highlighting}[]
\CommentTok{# Add variable called `error` to the object `polls` that contains the difference between d_hat and the actual difference on election day. Then make a plot of the error stratified by pollster.}
\NormalTok{polls }\OperatorTok\StringTok{ }\KeywordTok{mutate}\NormalTok{(}\DataTypeTok{error =}\NormalTok{ d_hat }\OperatorTok{-}\StringTok{ }\FloatTok{0.021}\NormalTok{) }\OperatorTok\StringTok{ }\KeywordTok{ggplot}\NormalTok{(}\KeywordTok{aes}\NormalTok{(}\DataTypeTok{x =}\NormalTok{ pollster, }\DataTypeTok{y =}\NormalTok{ error)) }\OperatorTok{+}\StringTok{ }\KeywordTok{geom_point}\NormalTok{() }\OperatorTok{+}\StringTok{ }\KeywordTok{theme}\NormalTok{(}\DataTypeTok{axis.text.x =} \KeywordTok{element_text}\NormalTok{(}\DataTypeTok{angle =} \DecValTok{90}\NormalTok{, }\DataTypeTok{hjust =} \DecValTok{1}\NormalTok{))}
\end{Highlighting}
\end{Shaded}

\includegraphics{03_-_Confidence_Intervals_and_P-Values_files/figure-latex/unnamed-chunk-8-1.pdf}

\subsection{\texorpdfstring{\textbf{Comparing to actual results by
pollster - multiple
polls}}{Comparing to actual results by pollster - multiple polls}}\label{comparing-to-actual-results-by-pollster---multiple-polls}

Remake the plot you made for the previous exercise, but only for
pollsters that took five or more polls.

You can use dplyr tools \texttt{group\_by} and \texttt{n} to group data
by a variable of interest and then count the number of observations in
the groups. The function \texttt{filter} filters data piped into it by
your specified condition.

For example:

\begin{verbatim}
data %>% group_by(variable_for_grouping) 
    %>% filter(n() >= 5)
\end{verbatim}

Instructions

\begin{itemize}
\tightlist
\item
  Define a new variable \texttt{errors} that contains the difference
  between the estimated difference between the proportion of voters and
  the actual difference on election day, 0.021.
\item
  Group the data by pollster using the \texttt{group\_by} function.
\item
  Filter the data by pollsters with 5 or more polls.
\item
  Use \texttt{ggplo}t to create the plot of errors by pollster.
\item
  Add a layer with the function \texttt{geom\_point}.
\end{itemize}

\begin{Shaded}
\begin{Highlighting}[]
\CommentTok{# The `polls` object has already been loaded. Examine it using the `head` function.}
\KeywordTok{head}\NormalTok{(polls)}
\end{Highlighting}
\end{Shaded}

\begin{verbatim}
##   state  startdate    enddate
## 1  U.S. 2016-11-03 2016-11-06
## 2  U.S. 2016-11-01 2016-11-07
## 3  U.S. 2016-11-02 2016-11-06
## 4  U.S. 2016-11-04 2016-11-07
## 5  U.S. 2016-11-03 2016-11-06
## 6  U.S. 2016-11-03 2016-11-06
##                                                     pollster grade
## 1                                   ABC News/Washington Post    A+
## 2                                    Google Consumer Surveys     B
## 3                                                      Ipsos    A-
## 4                                                     YouGov     B
## 5                                           Gravis Marketing    B-
## 6 Fox News/Anderson Robbins Research/Shaw & Company Research     A
##   samplesize population rawpoll_clinton rawpoll_trump rawpoll_johnson
## 1       2220         lv           47.00         43.00            4.00
## 2      26574         lv           38.03         35.69            5.46
## 3       2195         lv           42.00         39.00            6.00
## 4       3677         lv           45.00         41.00            5.00
## 5      16639         rv           47.00         43.00            3.00
## 6       1295         lv           48.00         44.00            3.00
##   rawpoll_mcmullin adjpoll_clinton adjpoll_trump adjpoll_johnson
## 1               NA        45.20163      41.72430        4.626221
## 2               NA        43.34557      41.21439        5.175792
## 3               NA        42.02638      38.81620        6.844734
## 4               NA        45.65676      40.92004        6.069454
## 5               NA        46.84089      42.33184        3.726098
## 6               NA        49.02208      43.95631        3.057876
##   adjpoll_mcmullin  d_hat
## 1               NA 0.0400
## 2               NA 0.0234
## 3               NA 0.0300
## 4               NA 0.0400
## 5               NA 0.0400
## 6               NA 0.0400
\end{verbatim}

\begin{Shaded}
\begin{Highlighting}[]
\CommentTok{# Add variable called `error` to the object `polls` that contains the difference between d_hat and the actual difference on election day. Then make a plot of the error stratified by pollster, but only for pollsters who took 5 or more polls.}
\NormalTok{polls }\OperatorTok\StringTok{ }\KeywordTok{mutate}\NormalTok{(}\DataTypeTok{error =}\NormalTok{ d_hat }\OperatorTok{-}\StringTok{ }\FloatTok{0.021}\NormalTok{) }\OperatorTok\StringTok{ }\KeywordTok{group_by}\NormalTok{(pollster) }\OperatorTok\StringTok{ }\KeywordTok{filter}\NormalTok{(}\KeywordTok{n}\NormalTok{() }\OperatorTok{>=}\StringTok{ }\DecValTok{5}\NormalTok{) }\OperatorTok\StringTok{ }\KeywordTok{ggplot}\NormalTok{(}\KeywordTok{aes}\NormalTok{(}\DataTypeTok{x =}\NormalTok{ pollster, }\DataTypeTok{y =}\NormalTok{ error)) }\OperatorTok{+}\StringTok{ }\KeywordTok{geom_point}\NormalTok{() }\OperatorTok{+}\StringTok{ }\KeywordTok{theme}\NormalTok{(}\DataTypeTok{axis.text.x =} \KeywordTok{element_text}\NormalTok{(}\DataTypeTok{angle =} \DecValTok{90}\NormalTok{, }\DataTypeTok{hjust =} \DecValTok{1}\NormalTok{))}
\end{Highlighting}
\end{Shaded}

\includegraphics{03_-_Confidence_Intervals_and_P-Values_files/figure-latex/unnamed-chunk-9-1.pdf}


\end{document}
